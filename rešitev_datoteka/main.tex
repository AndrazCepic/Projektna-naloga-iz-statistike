\documentclass{article}

\usepackage[slovene]{babel}
\usepackage[T1]{fontenc}
\usepackage[utf8]{inputenc}
\usepackage{listings}
\usepackage{amsmath}
\usepackage{float}
\usepackage{graphicx}
\usepackage{lmodern} 
\usepackage[a4paper, total={6in, 8in}]{geometry}

\title{Projektna naloga iz statistike}
\author{Andraž Čepič}
\date{2. 6. 2022}

\DeclareMathOperator{\std_err}{SE}
\DeclareMathOperator{\exp_val}{E}
\DeclareMathOperator{\variacija}{var}
\DeclareMathOperator{\verjetje}{L}
\DeclareMathOperator{\verjetnost}{P}
\DeclareMathOperator{\fisher}{FI}

\begin{document}

\maketitle

V projektu ves čas uporabljamo Python s paketi Pandas, NumPy, Jupyter in Matplotlib. Vsi programi, spisani v namen obdelave podatkov, se nahajajo v mapi \texttt{skripte}. 

\section*{Naloga 1}
V namen obdelave podatkov smo napisali Jupyter zvezek \texttt{kibergrad.ipynb}. Za začetek naložimo podatke iz datoteke \texttt{kibergrad.csv} v Pandas DataFrame objekt. 

\subsection*{(a)}
Izberemo enostavni slučanji vzorec velikosti $200$ s funkcijo \texttt{pandas.DataFrame.sample}. Če so
\begin{equation*}
    X_1, \ldots, X_{200}
\end{equation*}
števila otrok vsake od vzorčenih družin, je primerna ocena za povprečje enaka
\begin{equation*}
    \overline{X} = \frac{X_1 + \cdots + X_{200}}{200}.
\end{equation*}
Za naš specifičen vzorec dobimo oceno za povprečno število otrok v mestu Kibergrad:
\begin{equation*}
    \overline{X} = 0{,}925
\end{equation*}

\subsection*{(b)}
Ocena za standardno napako je podana s formulo
\begin{equation*}
    \widehat{\std_err}^2 = \frac{N-1}{N} \cdot \frac{1}{n(n-1)}\sum_{i=1}^{n}(X_i - \overline{X})^2,
\end{equation*}
kjer je $N$ velikost populacije in $n$ velikost enostavnega slučanega vzorca. V našem primeru je $N = 43.886$ in $n = 200$. Tako za naš vzorec dobimo
\begin{equation*}
    \widehat{\std_err} = 0{,}0808
\end{equation*}

Za enostavno slučajno vzorčenje so intervali zaupanja ocen povprečja oblike
\begin{equation*}
    \overline{X} - \widehat{\std_err} \cdot F^{-1}_{t}(1 - \frac{\alpha}{2}) < \mu < \overline{X} + \widehat{\std_err} \cdot F^{-1}_{t}(1 - \frac{\alpha}{2}),
\end{equation*}
kjer je $F_t$ komulativna funkcija Studentove $t$-porazdelitve z $n-1$ prostostnimi stopnjami in $\alpha = 0.05$ stopnja tveganja. V našem primeru dobimo interval zaupanja
\begin{equation*}
    0{,}7657 < \mu < 1{,}0843
\end{equation*}

\subsection*{(c)}
Pravo populacijsko povprečje se glasi
\begin{equation*}
    \mu = \frac{x_1 + \ldots + x_N}{N} = 0{,}9479.
\end{equation*}
Prava standardna napaka za enostavni slučanji vzorec velikosti $n = 200$ je
\begin{equation}
    \label{standardna_napaka_formula}
    \std_err^2 = \frac{N - n}{N - 1} \cdot \frac{\sigma^2}{n},
\end{equation}
kjer je $\sigma ^2$ variacija za celo populacijo. Za naše podatke je
\begin{equation*}
    \std_err = 0{,}0816.
\end{equation*}
Opazimo, da je ocena za povprečje malce manjša od pravega povprečja in ocena za standardno napako je prav tako malo manjša, vendar se razlikuje šele v tretji decimalki. Da, interval zaupanja pokrije populacijsko povprečje.

\subsection*{(d)}
Intervale zaupanja izračunamo na enak način, kot smo ga za prvi vzorec. Rezultati se nahajajo v mapi \texttt{rezultati}, in sicer v \texttt{intervali\char`_zaupanja.csv}. Naslednja slika prikazuje te intervale zaupanja in populacijsko povprečje:

\begin{figure}[H]
    \includegraphics[scale=0.4]{../rezultati/intervali_zaupanja_100_vzorcev_pop_povprecje.jpg}
\end{figure}

Izračunamo, da populacijsko povprečje pokrije $96$ intervalov zaupanja, oz. delež intervalov, ki pokrijejo populacijsko povprečje, je $0{,}96$.

\subsection*{(e)}
Če označimo $i$-to oceno povprečja iz $i$-tega vzorca z $\mu_i$ in z $\overline{\mu}$ označimo povprečje teh ocen, je potem ocena za varianco teh ocen enaka
\begin{equation*}
    \widehat{\sigma}^2 =  \frac{1}{100 - 1} \sum_{i=1}^{100} (\mu_i - \overline{\mu})^2.
\end{equation*}
Torej je standardni odklon enak
\begin{equation*}
    \widehat{\sigma} = 0{,}0776.
\end{equation*}
Prava standardna napaka za vzorec velikosti $200$ pa je
\begin{equation*}
    \std_err = 0{,}0816,
\end{equation*}
kar vemo že od prej. Opazimo, da je standardni odklon ocen povprečja manjši od standardne napake.

\subsection*{(f)}
Tedaj so rezultati o intervalih zaupanja shranjeni v datoteki \texttt{intervali\char`_zaupanja\char`_1.csv} v mapi \texttt{rezultati}. Grafično je v tem primeru
\begin{figure}[H]
    \includegraphics[scale=0.4]{../rezultati/intervali_zaupanja_100_vecjiih_vzorcev_pop_povprecje.jpg}
\end{figure}
Takoj opazimo, oziroma preverimo računsko, da tedaj vsi intervali zaupanja pokrijejo populacijsko povprečje. 

V tem primeru je standardni odklon ocen za povprečje enak
\begin{equation*}
    \widehat{\sigma}_1 = 0{,}0399.
\end{equation*}
Prava standardna napaka za vzorec velikosti $800$ pa je
\begin{equation*}
    \std_err_1 = 0{,}0405.
\end{equation*}

Iz formule \eqref{standardna_napaka_formula} je očitno, da je za večje velikosti vzorcev $n$ standardna napaka manjša, zato nas ne preseneča, da je prava stantardna napaka za vzorce velikosti $800$ precej manjša od tiste za vzorce velikosti $200$. Zanimivo je, da se standardni odklon ocen za povprečje v obeh primerih precej ujema s pravimi napakami, torej je standardni odklon ocen za vzorce velikosti $800$ približno pol manjši od tistega, ki pride iz vzorcev velikosti $200$. TODO: pojasni do konca.

\section*{Naloga 2}
Obdelava podatkov je v Jupyter datoteki \texttt{naloga\char`_2.ipynb}.

\subsection*{(a)}
Naj bo $X$ spremenljivka z dano porazdelitivjo, odvisno od parametra $\theta$. Tedaj je 
\begin{align*}
    \exp_val (X) &= \frac{1}{3}\theta + \frac{4}{3}(1 - \theta) + (1 - \theta) \\
    &= -2 \theta + \frac{7}{3}.
\end{align*}
Torej je
\begin{equation*}
    \theta = \frac{7}{6} - \frac{1}{2} \exp_val (X),
\end{equation*}
zato je
\begin{equation*}
    \widehat{\theta} := \frac{7}{6} - \frac{1}{2} \overline{X}
\end{equation*}
cenilka za parameter $\theta$ preko ocene prvega momenta $\overline{X}$. Ocena $\overline{X}$ za $\exp_val(X)$ je nepristranska in pričakovana vrednost je linearna, torej je $\widehat{\theta}$ nepristranska cenilka za $\theta$. Po naših opažanjih dobimo oceno
\begin{equation*}
    \widehat{\theta} = 0{,}3667.
\end{equation*}

\subsection*{(b)}
Srednja kvadratična napaka te ocene je
\begin{equation*}
    \std_err^2 = \exp_val \left( \left(\widehat{\theta} - \theta \right)^2 \right)= \variacija \left(\widehat{\theta} \right) = \frac{\sigma^2}{10},
\end{equation*}
zato je nepristranska ocena za kvadrat standardne napake pri metodi momentov enaka
\begin{equation*}
    \widehat{\std_err}^2 = \frac{\widehat{\sigma}_{+}^2}{10} = \frac{1}{10}\cdot \frac{1}{10-1} \sum_{i=1}^{10}(X_i - \overline{X})^2
\end{equation*}
in pri teh konkretnih opažanjih ocena za standardno napako znaša
\begin{equation*}
    \widehat{\std_err} = 0{,}3399.
\end{equation*}

\subsection*{(c)}
Če so opažene vrednosti označene po vrsti z $x_1, \ldots, x_{10}$ in slučajne spremenljivke teh opažanj z $X_1, \ldots, X_{10}$, je verjetje za naša opažanja zaradi neodvisnosti enako
\begin{align*}
    \verjetje(\theta; X) &= \verjetnost(X_1 = x_1) \cdots \verjetnost(X_{10} = x_{10}) \\
    &= \frac{2^6}{3^{10}}\cdot \theta^4 (1- \theta)^6.
\end{align*}
Iščemo maksimum verjetja. Lažje je opravljati z logaritmom, zato definiramo
\begin{equation*}
    l(\theta; X) = \log(\verjetje(\theta; X)) =  \log \left(\frac{2^6}{3^{10}} \right) + 4\log \theta + 6\log(1- \theta).
\end{equation*}
Logaritem je monotona funkcija, zato bo maksimum $\verjetje(\theta; X)$ dosežen ob istem $\theta$ kot za $l(\theta; X)$. Odvod $l$ je potem
\begin{equation*}
    \frac{\partial l}{\theta} = \frac{4}{\theta} - \frac{6}{1 - \theta}.
\end{equation*}
Rešujemo torej enačbo
\begin{equation*}
    \frac{4}{\theta} - \frac{6}{1 - \theta} = 0,
\end{equation*}
zato je
\begin{align*}
    \frac{4}{\theta} &= \frac{6}{1-\theta} \\
    6 \theta &= 4 - 4 \theta \\
    5 \theta &= 2 \\
    \theta &= \frac{2}{5}.
\end{align*}
To je edini lokalni ekstrem na notranjosti intervala $[0, 1]$, hkrati pa je $L$ na robovih intervala enak $0$ in $L(\frac{2}{5}; X) > 0$, zato je to res globalni maksimum verjetja. Za oceno parametra $\theta$ po metodi največjega verjetja torej vzamemo
\begin{equation*}
    \widehat{\theta} = \frac{2}{5} = 0{,}4.
\end{equation*}

\subsection*{(d)}
Za oceno standardne napake te ocene uporabimo Fisherjevo informacijo
\begin{equation*}
    \fisher(\theta) = - \exp_val \left(  \frac{\partial^2 l}{\partial \theta^2} (\theta; X)\right).
\end{equation*}
Ocena za standardno napako se zdaj glasi
\begin{equation*}
    \widehat{\std_err} = \frac{1}{\sqrt{\fisher(\theta)}}.
\end{equation*}
V našem primeru je
\begin{equation*}
    \frac{\partial^2 l}{\partial \theta^2} (\theta; X) = -\frac{4}{\theta^2} - \frac{6}{(1-\theta)^2},
\end{equation*}
zato je
\begin{equation*}
    \fisher(\theta) = \frac{4}{\theta^2} + \frac{6}{(1-\theta)^2}
\end{equation*}
in končno v našem primeru $\theta = \frac{2}{5}$ dobimo oceno za standardno napako:
\begin{equation*}
    \widehat{\std_err} = 0{,}1549.
\end{equation*}

\subsection*{(e)}
Naj bo $f_{\theta}$ gostota porazdelitve spremenljivke $\theta$. Na intervalu $[0,1]$ je potem konstantno enaka $1$, drugje pa $0$. Naj bo $H$ dogodek, da so se zgodila neodvisna opažanja $X_1 = x_1, \ldots, X_{10} = x_{10}$ kot podano. Za pogojno gostoto $f_{\theta | H}$ ob opaženem dogodku $H$ uporabimo Bayesovo formulo za mešane porazdelitve:
\begin{equation*}
    f_{\theta | H} (t) = \frac{\verjetnost(H | \theta = t) \cdot f_{\theta}(t)}{\verjetnost(H)},
\end{equation*}
kjer je verjetnost dogodka $H$ enaka
\begin{equation*}
    \verjetnost(H) = \int_0^1 \verjetnost(H | \theta = t) f_{\theta} (t) dt.
\end{equation*}
Vemo, da je
\begin{equation*}
    \verjetnost(H | \theta = t) = \frac{2^6}{3^{10}} t^4 (1-t)^6,
\end{equation*}
torej je
\begin{equation*}
    \verjetnost(H) = \int_0^1 \frac{2^6}{3^{10}} t^4(1-t)^6 dt = \frac{2^6}{3^{10}} B(5, 7).
\end{equation*}
Sklepamo, da je
\begin{equation*}
    f_{\theta | H}(t) = \frac{1}{B(5, 7)} t^4(1-t)^6,
\end{equation*}
kjer je $t \in [0, 1]$, drugje je pa enaka $0$. Ugotovili smo, da je aposteriorna porazdelitev $\theta | H$ porazdeljena z beta porazdelitvijo $B(5, 7)$.

\end{document}